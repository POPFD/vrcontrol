This plugin allows you to use a Python 3.\+x script to process images received by mjpg-\/streamer. This has been tested with Python 3.\+4.

To run a Python 3.\+x script, you can do the following\+: \begin{DoxyVerb}mjpg_streamer -i "input_opencv.so --filter cvfilter_py.so --fargs path/to/filter.py"
\end{DoxyVerb}


\subsection*{Filter script }

Your script M\+U\+S\+T define a function called 'init\+\_\+filter', that takes zero arguments and returns a single callable. This returned callable must take a single argument (a numpy array), and returns a single object (a numpy array). A simple example follows\+:

```

def filter\+\_\+fn(img)\+: ''' \+:param img\+: A numpy array representing the input image \+:returns\+: A numpy array to send to the mjpg-\/streamer output plugin ''' return img

def init\+\_\+filter()\+: return filter\+\_\+fn

```

For a more complex example, see the included example\+\_\+filter.\+py

\subsection*{Known Issues }

When mjpg-\/streamer is terminated, a {\ttfamily Key\+Error} is raised in the threading module. While annoying, it's harmless. Most likely it happens because the python interpreter is destroyed on the wrong thread.

\subsection*{T\+O\+D\+O }

After going through all of this effort to create the code for this module, I bet it can be done a lot simpler via cython.

\subsection*{Authors }

Dustin Spicuzza (\href{mailto:dustin@virtualroadside.com}{\tt dustin@virtualroadside.\+com}) 